Turnover box system, the term is unfamiliar to most of us, yet it tightly connects with our daily life. Narrowly speaking, turnover box system aims to track the delivery of boxes among the partners in a certain supply chain. Those turnover boxes are containers of fruits, vegetables,
agricultural product, etc. In this chapter, I'll present the State-of-the-art of the supply chain, and its demand, requirement and problem. Also transit boxes as foundation of the logistic chain, how the blockchain system could be leveraged to boost the process.

\section{Supply Chain Management(SCM)} 
Ixxxxxxxxxxxxxxxxxxxxxxxxxxxxxxxxxxxxxxxxxxxxxxxxxxxxxxxxxxxxxxxxxxxxxxxxxxxxxxxxxxxxxxxxxxxxxxxxxxxxxxxxxxxxxxxxx

\begin{itemize}
	\item \textbf{Connected}\\
Ixxxxxxxxxxxxxxxxxxxxxxxxxxxxxxxxxxxxxxxxxxxxxxxxxxxxxxxxxxxxxxxxxxxxxxxxxxxxxxxxxxxxxxxxxxxxxxxxxxxxxxxxxxxxxxxxx
	\item \textbf{ Collaborative}\\
Ixxxxxxxxxxxxxxxxxxxxxxxxxxxxxxxxxxxxxxxxxxxxxxxxxxxxxxxxxxxxxxxxxxxxxxxxxxxxxxxxxxxxxxxxxxxxxxxxxxxxxxxxxxxxxxxxx
	\item \textbf{Cyberaware}\\Ixxxxxxxxxxxxxxxxxxxxxxxxxxxxxxxxxxxxxxxxxxxxxxxxxxxxxxxxxxxxxxxxxxxxxxxxxxxxxxxxxxxxxxxxxxxxxxxxxxxxxxxxxxxxxxxxx
	\item \textbf{Cognitively enabled}\\
	The AI platform becomes the modern supply chain's control tower by
	collating, coordinating, and conducting decisions and next best actions across the chain in an
	automated and timely way. Certain exceptions would require human
	intervention, but most of the supply chain would be automated and self-learning.
	\item \textbf{Comprehensive}\\
	Analytics capabilities must be scaled with data and in real time and insights must be comprehensive and fast.
\end{itemize}

\section{Logistics}
Logistics, the part of supply chain management, is the process of planning, implementing and controlling procedures for the efficient and effective transportation and storage of goods including services and related information from the point of origin to the point of consumption for the purpose of conforming to customer requirements and includes inbound, outbound, internal and external movements.

In order to realize the digitalized and smart supply chain, the modern Logistics, as the very essential part of the SCM, cannot come without roboticized, automatized process, telematics, Big Data, Cloud-based system.  Especially, the transport tool, that is, Turnover box draws the attention from companies, which are engaged in optimized the logistic process.

\textbf{Turnover Box}
- It is also called logistics box or transit box (hereinafter called the 'Box' also). In the Information age, it represents not purely the container of the goods, but integrates the relevant states, identity, channel, etc information.  

Thus Lufthansa Industry Solution proposed to setup a cloud-based platform integrated with logistics boxes, so that the clients can better track, check the goods, faster routing planning help to reduce the cost.\cite{lufthansa}

\section{Turnover Box System}
Why we focus on designing, developing and evaluating the Turnover Box System based on blockchain? Not only because several exploration in this specific field expands and even successful use cases come out from enterprises like Lufthansa, IBM, Deloitte. But also inspired by a client, who operates a company providing Boxes to the participants (For example, agri-food suppliers, goods distributors, product retailers) in certain supply chains, wanted an efficient, fault-tolerant, integrated system based on blockchain to help operate, manipulate and trace the Boxes and the cash flow. 

Thus this thesis, standing from the perspective of a Turnover Box operator, will depict, develop a digitalized, effective, easy access blockchain-based system, and evaluate whether the blockchain platform fits well in this scenarios.

\subsection{Challenges}
Digitalization in Business is changing the way we conduct business, the way we communicate, transact and interact with customers. And they tightly depend on their supply chain. As clients demand more transparency, the complexity of supply chains increases. An effective and inexpensive way to trace each material used in the final product is important in building confidence with increasingly environmental and socially conscious consumers. 
In sum, we list out the most concrete challenges when building such a versatile system:

\begin{itemize}
	\item \textbf{Transaction Throughput}\\
		The latest statistic shows that in recent 5 years every German consumes approx. 70kg fruits and approx. 98kg vegetables per year\cite{obst}. If we only take the city Düsseldorf with 660,000 people as an example, and only consider the consumption of fruits and vegetables, the whole city may need almost 1,1000-1,2000 Boxes (most common Boxes with 25kg load capacity) per day to deliver. So if we count in other agricultural products, and larger supply chain network in other cities, the pressure is obvious. Such huge volume of transactions and data must be proceeded efficiently.
		
	\item \textbf{Scalability}\\
		For a striving company, its business extends rather fast. While they add new partners, nodes and peers into their network, they also don't want the expansion obviously affect the user experience. And it also helps maintain the longevity of the blockchain (which is also essential for a supply chain). In the KMPG's latest report Demand-driven supply chain, they ranked the scalability as the third place.
	
	\item \textbf{Confidentiality}\\
		In real commercial world, the transaction confidentiality is sometime essential and useful. e.g. the operator want to take different price strategies among its clients, or in order to protect the data privacy of the product suppliers, the data confidentiality of the transaction really matters.
	
	\item \textbf{Robustness}\\
		For such a supply chain with various parties and transaction especially concerning payment, it is required that the system tough, dependable enough. The single-failure(usually happened in centralized system) should be avoided, to say least, the broken down or defected system(ledger) should be recovered easily. And it's also the reason we adopt the blockchain to track the flow.
		
	\item \textbf{Data Transparency}\\
		Not necessarily suitable for our Turnover Box System, but in Supply Chain becomes an increasingly significant elements. Specially in food supply chain, the source of raw material, producers, transport methods, expiration date are the focus of management to ensure food safety. Accordingly it helps to build more trustful cooperation among partners.    
\end{itemize}


\section{Key advantages of blockchain over traditional systems}
We chose blockchain technologies  as the basis to rebuild our digital supply chain system, because we did realize the challenges we faced, and recognized the features of blockchain which is immutable, transparent, and redefines trust, enables secure, fast, trustworthy, and transparent solutions that can be public or private.Those features surpass many traditional centralized ledgers or systems. The following lists shows the details:

\begin{itemize}
	\item \textbf{More Autonomy}
	
	 In blockchain-based system, every node plays roles as both of clients and servers. Nodes can submit the transactions at anytime. And the acceptance of the transaction depends on the counterparties and consensus algorithm. While the nodes in the centralized systems (traditional client-server mode) have to submit the proposal to the central server waiting to be proceeded. The way of submit transaction is much more autonomous.
	 
	\item \textbf{Higher Throughput} 
	
	As explained above, each node is both a client and a server. So many transactions submission and processing can run in parallel, which reduces the congestion in the network, improve the overall throughput rate.
	 
	\item \textbf{Automation}
	
	Smart contracts transfer the business logic into the programmed code. When similar business happens, the pre-set program will be triggered to proceed a set of actions.
	
	\item \textbf{Faster Transaction Process}
	
	Distributed design helps to improve the use rates of computing resources, which helps to reduce processing time
	
	\item \textbf{Robustness}
	
	Blockchain system basically avoids the "single failure" problem, which is the pain point of the tradition systems.
	
	\item \textbf{Immutability}
	
	In blockchain system, each entity must have assurance that their copy of the ledger is identical to other participants. This is the only way it can assure itself that the transactions it participates in are valid and unique. Malicious user must change record on all the ledger in the network simultaneously if it wants to tamper the ledger. 
\end{itemize}

\begin{comment}
	\item IBM cooperates with Walmart aiming to create a fully transparent food safe system \cite{walmart}.
	\item Oliver Wyman provides an End-To-End Blockchain enabled Supply Chain.
\end{comment} 


%(blockchain ubiquity)
Thus blockchain could be used to address inefficiency, lacking transparency and visibility, etc in current systems. Blockchain’s distributed ledger offers a means for exchanging assets in an open, secure protocol, which has interesting. To be more specify, how those issues are settle and how well they would perform will be our focus in this thesis. 

Thus it is promising but also challenging to build supply chain capabilities with the aid of blockchain which can result in greater levels of performance. 


