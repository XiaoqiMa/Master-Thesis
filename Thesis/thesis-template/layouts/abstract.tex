\thispagestyle{plain}
\topskip0pt
\vspace*{\fill}
\begin{center}
	\Large
	\textbf{Abstract}
\end{center}

\vspace{1cm}
Blockchain is a distributed or shared ledger that holds records of digital transactions in such a way that makes them accessible and visible to multiple participants in a network, while keeping them secure. In recent years, we have witnessed an increasing number of forward-thinking enterprises, organizations explore and deploy the blockchain technology in a wide variety of fields, aiming to cope with the newly emerging problems. 

Supply chain one of those fields which encountered challenges from new era. Today it has an increase requirement for continually ensuring the quality, delivery and availability of supply while controlling costs. Lack of visibility and transparency is the greatest hurdle in achieving the supply chain organizations' objectives. Thus under great pressure, the blockchain-based solution has been proposed to assist.

The goal of this work is to prove a detailed description of how blockchain technology can be applied to solve the pain point in supply chain. 
Since Turnover Box plays especially a critical role in supply chain, it will be chosen as our test bed to deploy blockchain technology. 

This thesis is structured as follow: it started with the origin of blockchain technology (the birth of Bitcoin), and its development. We present the most ubiquitous blockchain platforms and wisely to pick out the most suitable platforms to develop the system based on some reasonable criteria. Later after the development, we will evaluate the system, in order to find the optimal solution for this type of supply chain.

    \textbf{Keywords}: \textit{Blockchain; Supply Chain; Bitcoin;  Transparency; Turnover Box; Visibility}
\vspace*{\fill}